%        File: summary.tex
%     Created: Sun Feb 15 11:00 PM 2015 C
% Last Change: Sun Feb 15 11:00 PM 2015 C
%
\documentclass[a4paper]{report}

\usepackage{amsmath}
\usepackage{amsthm}

\begin{document}
\chapter{Probability}

\begin{section}{Sample Spaces and Events}

\paragraph{The Sample Space of an Experiment:} The \textbf{sample space} of an
experiment, denoted by $S$, is the set of all possible outcomes of that
experiment.

\paragraph{Events:} An \textbf{event} is any collection (subset) of outcomes
contained in the sample space $S$. An event is said to be \textbf{simple} if it
consists of exactly one outcome and \textbf{compound} if it consists of more
than one outcome.

\paragraph{Some Relations from Set Theory:}

\begin{enumerate}
  \item The \textbf{union} of two events $A$ and $B$, denoted by $A\cup B$ and
    read ``$A$ or $B$'' is the event consisting of all outcomes that are
    \textit{either in $A$ or in $B$ or in both events} (so that the union
    includes outcomes for which both $A$ and $B$ occur as well as the outcomes
    for which exactly one occurs) - that is, all outcomes in at least one of
    the events.
  \item The \textbf{intersection} of two events $A$ and $B$, denoted by $A \cap
    B$ and read `` \textit{$A$ and $B$} '' is the event consisting of all
    outcomes that are in \textit{both $A$ and $B$}.
  \item The \textbf{complement} of an event $A$, denoted by $A'$, is the set of
    all outcomes in $S$ that are not contained in $A$.
\end{enumerate}

\paragraph{Mutually exlusive events:}

When $A$ and $B$ have no outcomes in common, they are said to be
\textbf{disjoint} or \textbf{mutually exclusive} events.  Mathematicians write
this compactly as $A \cup B = \emptyset$ where $\emptyset$ denotes the event
consisting of no outcomes whatsoever (the ``null'' or ``empty'' event).

\end{section}
\begin{section}{Axioms, Interpretations, and Properties of Probability}

\paragraph{Basic properties of probability:}
\begin{enumerate}
  \item For any event $A, P(A) \geq 0$.
  \item $P(S) = 1$.
  \item If $A_1, A_2, A_3, \cdots $ is an infinite collection of disjoint events, then \[P(A_1 \cup A_2 \cup \cdots) = \sum\limits_{i=1}^\infty P(A_i).\]
\end{enumerate}

\begin{subsection}{Proposition}

\quote{$P(\emptyset) = 0$ where $\emptyset$ is the null event. This in turn
  implies that the property contained in Axiom 3 is valid for a \textit{finite}
  collection of events.\footnote{Proof page 57.}}

\end{subsection}
\begin{subsection}{Proposition}
\quote{For any event $A, P(A) = 1 - P(A')$.\footnote{Proof page 60.}}
\end{subsection}
\begin{subsection}{Proposition}

\quote{For any event $A, P(A) \leq 1$.}
\end{subsection}
\begin{subsection}{Proposition}
\quote{
  For any events $A$ and $B$,
  \begin{align*}
    P(A\cup B) &= P(A) + P(B) - P(A\cap B).\footnote{Proof page 61}\\
    P(A \cup B \cup C) &= P(A) + P(B) + P(C) - P(A \cap B) - P(A \cap C)  \\
                       &- P(A \cap C) - P(B \cap C) + P(A \cap B \cap C)
  \end{align*}
}
\end{subsection}
\end{section}
\begin{section}{Counting Techniques}

\paragraph{The Product Rule for Ordered Pairs}

\begin{subsection}{Proposition}
\quote{
  If the first element or object of an ordered pair can be selected in $n_1$
  ways, and for each of these $n_1$ ways the second element of the pair can be
  selected in $n_2$ ways, then the number of pairs is $n_1n_2$.
}
\end{subsection}

\paragraph{Product Rule for $K$-Tuples:}

\quote{
  Suppose a set consists of ordered collections of $k$ elements ($k$-tuples)
  and that there are $n_1$ possible choices for the first element; for each
  choice of the first element, there are $n_2$ possible choices of the second
  element;\dots; for each possible of the first $k-1$ elements, there are $n_k$
  choices for $k$th element. Then there are $n_1n_2\cdot\dots\cdot n_k$
  possible $k$-tuples.  
}

\paragraph{Permutations:}

\quote{
  Any ordered sequence $k$ objects taken from a set of $n$ distinct objects is called a \textbf{permutation} of size $k$ of the objects.
  The numbers of permutations of size $k$ that can be constructed from the $n$ objects is denoted $P_{k, n}$.
}
\end{section}
\end{document}


